\documentclass{article}

\usepackage{amsmath}
\usepackage[a4paper, margin=.5in]{geometry}
\usepackage{amsfonts}

\author{Luke A.C.A. Rieff}
\title{Forward and Inverse kinematics for Robotic Rhino}
\date{9-12-2022}

\begin{document}
    \maketitle

    \section{Forward Kinematics}

    Before we start creating the kinematic model for one of the robotic legs, we're going to define
    the rotation matrices for rotations around the $x$, $y$ and $z$ axis.

    \begin{equation}
        R_x(\theta) =
        \begin{bmatrix}
            1 & 0 & 0 & 0 \\
            0 & \cos{\theta} & - \sin{\theta} & 0 \\
            0 & \sin{\theta} & \cos{\theta} & 0 \\
            0 & 0 & 0 & 1
        \end{bmatrix}
    \end{equation}

    \begin{equation}
        R_y(\theta) =
        \begin{bmatrix}
            \cos{\theta} & 0 & - \sin{\theta} & 0 \\
            0 & 1 & 0 & 0 \\
            - \sin{\theta} & 0 & \cos{\theta} & 0 \\
            0 & 0 & 0 & 1
        \end{bmatrix}
    \end{equation}

    \begin{equation}
        R_z(\theta) =
        \begin{bmatrix}
            \cos{\theta} & - \sin{\theta} & 0 & 0 \\
            \sin{\theta} & \cos{\theta} & 0 & 0 \\
            0 & 0 & 1 & 0 \\
            0 & 0 & 0 & 1
        \end{bmatrix}
    \end{equation}

    Also we define the translation matrices for translating a point in the $x$, $y$ and $z$ axis.

    \begin{equation}
        T_x(x) =
        \begin{bmatrix}
            1 & 0 & 0 & x \\
            0 & 1 & 0 & 0 \\
            0 & 0 & 1 & 0 \\
            0 & 0 & 0 & 1
        \end{bmatrix},
        T_y(y) =
        \begin{bmatrix}
            1 & 0 & 0 & 0 \\
            0 & 1 & 0 & y \\
            0 & 0 & 1 & 0 \\
            0 & 0 & 0 & 1
        \end{bmatrix},
        T_z(z) =
        \begin{bmatrix}
            1 & 0 & 0 & 0 \\
            0 & 1 & 0 & 0 \\
            0 & 0 & 1 & z \\
            0 & 0 & 0 & 1
        \end{bmatrix},
        T(x, y, z) = \begin{bmatrix}
            1 & 0 & 0 & x \\
            0 & 1 & 0 & y \\
            0 & 0 & 1 & z \\
            0 & 0 & 0 & 1
        \end{bmatrix}
    \end{equation}

    The torso of the robotic rhino has the width $w$ and the length $h$ and has
    an origin at the vector $O_{\mathrm{torso}}\ \epsilon \ \mathbb{R}^4$.
    For its orientation in space we will use the euler angles yaw $\alpha$,
    pitch $\beta$ and roll $\gamma$.
    With these angles, the origin has its own
    transformation matrix $T_{\mathrm{torso}}$ which will transform all the origins of the legs.

    \begin{equation}
        M_{\mathrm{torso}} = T(O_{\mathrm{torso}}) R_z(\gamma) R_y(\beta) R_x(\alpha)
    \end{equation}

    Each of the four legs has its own origin $O_{\mathrm{side}}$ which is attached to one of the
    corners of the square torso.

    \begin{equation}
        O_{\mathrm{back-left}} = M_{\mathrm{torso}}
        \begin{bmatrix}
            - \frac{w}{2} \\
            - \frac{h}{2} \\
            0 \\
            0
        \end{bmatrix},
        O_{\mathrm{back-right}} = M_{\mathrm{torso}}
        \begin{bmatrix}
            \frac{w}{2} \\
            - \frac{h}{2} \\
            0 \\
            0
        \end{bmatrix}
    \end{equation}


    \begin{equation}
        O_{\mathrm{front-left}} = M_{\mathrm{torso}}
        \begin{bmatrix}
            - \frac{w}{2} \\
            \frac{h}{2} \\
            0 \\
            0
        \end{bmatrix},
        O_{\mathrm{front-right}} = M_{\mathrm{torso}}
        \begin{bmatrix}
            \frac{w}{2} \\
            \frac{h}{2} \\
            0 \\
            0
        \end{bmatrix}
    \end{equation}

\end{document}